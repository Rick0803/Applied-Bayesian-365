% Options for packages loaded elsewhere
\PassOptionsToPackage{unicode}{hyperref}
\PassOptionsToPackage{hyphens}{url}
%
\documentclass[
]{article}
\usepackage{amsmath,amssymb}
\usepackage{lmodern}
\usepackage{ifxetex,ifluatex}
\ifnum 0\ifxetex 1\fi\ifluatex 1\fi=0 % if pdftex
  \usepackage[T1]{fontenc}
  \usepackage[utf8]{inputenc}
  \usepackage{textcomp} % provide euro and other symbols
\else % if luatex or xetex
  \usepackage{unicode-math}
  \defaultfontfeatures{Scale=MatchLowercase}
  \defaultfontfeatures[\rmfamily]{Ligatures=TeX,Scale=1}
\fi
% Use upquote if available, for straight quotes in verbatim environments
\IfFileExists{upquote.sty}{\usepackage{upquote}}{}
\IfFileExists{microtype.sty}{% use microtype if available
  \usepackage[]{microtype}
  \UseMicrotypeSet[protrusion]{basicmath} % disable protrusion for tt fonts
}{}
\makeatletter
\@ifundefined{KOMAClassName}{% if non-KOMA class
  \IfFileExists{parskip.sty}{%
    \usepackage{parskip}
  }{% else
    \setlength{\parindent}{0pt}
    \setlength{\parskip}{6pt plus 2pt minus 1pt}}
}{% if KOMA class
  \KOMAoptions{parskip=half}}
\makeatother
\usepackage{xcolor}
\IfFileExists{xurl.sty}{\usepackage{xurl}}{} % add URL line breaks if available
\IfFileExists{bookmark.sty}{\usepackage{bookmark}}{\usepackage{hyperref}}
\hypersetup{
  pdftitle={STA365\_homework2\_code},
  pdfauthor={Ruike Xu},
  hidelinks,
  pdfcreator={LaTeX via pandoc}}
\urlstyle{same} % disable monospaced font for URLs
\usepackage[margin=1in]{geometry}
\usepackage{color}
\usepackage{fancyvrb}
\newcommand{\VerbBar}{|}
\newcommand{\VERB}{\Verb[commandchars=\\\{\}]}
\DefineVerbatimEnvironment{Highlighting}{Verbatim}{commandchars=\\\{\}}
% Add ',fontsize=\small' for more characters per line
\usepackage{framed}
\definecolor{shadecolor}{RGB}{248,248,248}
\newenvironment{Shaded}{\begin{snugshade}}{\end{snugshade}}
\newcommand{\AlertTok}[1]{\textcolor[rgb]{0.94,0.16,0.16}{#1}}
\newcommand{\AnnotationTok}[1]{\textcolor[rgb]{0.56,0.35,0.01}{\textbf{\textit{#1}}}}
\newcommand{\AttributeTok}[1]{\textcolor[rgb]{0.77,0.63,0.00}{#1}}
\newcommand{\BaseNTok}[1]{\textcolor[rgb]{0.00,0.00,0.81}{#1}}
\newcommand{\BuiltInTok}[1]{#1}
\newcommand{\CharTok}[1]{\textcolor[rgb]{0.31,0.60,0.02}{#1}}
\newcommand{\CommentTok}[1]{\textcolor[rgb]{0.56,0.35,0.01}{\textit{#1}}}
\newcommand{\CommentVarTok}[1]{\textcolor[rgb]{0.56,0.35,0.01}{\textbf{\textit{#1}}}}
\newcommand{\ConstantTok}[1]{\textcolor[rgb]{0.00,0.00,0.00}{#1}}
\newcommand{\ControlFlowTok}[1]{\textcolor[rgb]{0.13,0.29,0.53}{\textbf{#1}}}
\newcommand{\DataTypeTok}[1]{\textcolor[rgb]{0.13,0.29,0.53}{#1}}
\newcommand{\DecValTok}[1]{\textcolor[rgb]{0.00,0.00,0.81}{#1}}
\newcommand{\DocumentationTok}[1]{\textcolor[rgb]{0.56,0.35,0.01}{\textbf{\textit{#1}}}}
\newcommand{\ErrorTok}[1]{\textcolor[rgb]{0.64,0.00,0.00}{\textbf{#1}}}
\newcommand{\ExtensionTok}[1]{#1}
\newcommand{\FloatTok}[1]{\textcolor[rgb]{0.00,0.00,0.81}{#1}}
\newcommand{\FunctionTok}[1]{\textcolor[rgb]{0.00,0.00,0.00}{#1}}
\newcommand{\ImportTok}[1]{#1}
\newcommand{\InformationTok}[1]{\textcolor[rgb]{0.56,0.35,0.01}{\textbf{\textit{#1}}}}
\newcommand{\KeywordTok}[1]{\textcolor[rgb]{0.13,0.29,0.53}{\textbf{#1}}}
\newcommand{\NormalTok}[1]{#1}
\newcommand{\OperatorTok}[1]{\textcolor[rgb]{0.81,0.36,0.00}{\textbf{#1}}}
\newcommand{\OtherTok}[1]{\textcolor[rgb]{0.56,0.35,0.01}{#1}}
\newcommand{\PreprocessorTok}[1]{\textcolor[rgb]{0.56,0.35,0.01}{\textit{#1}}}
\newcommand{\RegionMarkerTok}[1]{#1}
\newcommand{\SpecialCharTok}[1]{\textcolor[rgb]{0.00,0.00,0.00}{#1}}
\newcommand{\SpecialStringTok}[1]{\textcolor[rgb]{0.31,0.60,0.02}{#1}}
\newcommand{\StringTok}[1]{\textcolor[rgb]{0.31,0.60,0.02}{#1}}
\newcommand{\VariableTok}[1]{\textcolor[rgb]{0.00,0.00,0.00}{#1}}
\newcommand{\VerbatimStringTok}[1]{\textcolor[rgb]{0.31,0.60,0.02}{#1}}
\newcommand{\WarningTok}[1]{\textcolor[rgb]{0.56,0.35,0.01}{\textbf{\textit{#1}}}}
\usepackage{graphicx}
\makeatletter
\def\maxwidth{\ifdim\Gin@nat@width>\linewidth\linewidth\else\Gin@nat@width\fi}
\def\maxheight{\ifdim\Gin@nat@height>\textheight\textheight\else\Gin@nat@height\fi}
\makeatother
% Scale images if necessary, so that they will not overflow the page
% margins by default, and it is still possible to overwrite the defaults
% using explicit options in \includegraphics[width, height, ...]{}
\setkeys{Gin}{width=\maxwidth,height=\maxheight,keepaspectratio}
% Set default figure placement to htbp
\makeatletter
\def\fps@figure{htbp}
\makeatother
\setlength{\emergencystretch}{3em} % prevent overfull lines
\providecommand{\tightlist}{%
  \setlength{\itemsep}{0pt}\setlength{\parskip}{0pt}}
\setcounter{secnumdepth}{-\maxdimen} % remove section numbering
\ifluatex
  \usepackage{selnolig}  % disable illegal ligatures
\fi

\title{STA365\_homework2\_code}
\author{Ruike Xu}
\date{3/29/2022}

\begin{document}
\maketitle

\hypertarget{question-2}{%
\subsection{Question 2}\label{question-2}}

\begin{Shaded}
\begin{Highlighting}[]
\CommentTok{\# install.packages("rjags")}

\FunctionTok{library}\NormalTok{(R2jags)}
\end{Highlighting}
\end{Shaded}

\begin{verbatim}
## Warning: package 'R2jags' was built under R version 4.1.3
\end{verbatim}

\begin{verbatim}
## Loading required package: rjags
\end{verbatim}

\begin{verbatim}
## Warning: package 'rjags' was built under R version 4.1.3
\end{verbatim}

\begin{verbatim}
## Loading required package: coda
\end{verbatim}

\begin{verbatim}
## Linked to JAGS 4.3.0
\end{verbatim}

\begin{verbatim}
## Loaded modules: basemod,bugs
\end{verbatim}

\begin{verbatim}
## 
## Attaching package: 'R2jags'
\end{verbatim}

\begin{verbatim}
## The following object is masked from 'package:coda':
## 
##     traceplot
\end{verbatim}

\begin{Shaded}
\begin{Highlighting}[]
\FunctionTok{library}\NormalTok{(lattice)}
\end{Highlighting}
\end{Shaded}

\[ Liberal:\ \ \tau(1-\theta)+\theta(1-\lambda) \]
\[ Republician(\psi):\ \ \lambda\theta + (1-\tau)(1-\theta) \]

\begin{Shaded}
\begin{Highlighting}[]
\CommentTok{\# Model construction with likelihood function and prior}
\NormalTok{model.JAGS }\OtherTok{=} \ControlFlowTok{function}\NormalTok{()\{}
\NormalTok{    y }\SpecialCharTok{\textasciitilde{}} \FunctionTok{dbinom}\NormalTok{(psi, n)}
\NormalTok{    theta }\SpecialCharTok{\textasciitilde{}} \FunctionTok{dbeta}\NormalTok{(}\DecValTok{3}\NormalTok{, }\DecValTok{7}\NormalTok{)}
\NormalTok{    tau }\SpecialCharTok{\textasciitilde{}} \FunctionTok{dbeta}\NormalTok{(}\DecValTok{6}\NormalTok{, }\DecValTok{4}\NormalTok{)}
\NormalTok{    lambda }\SpecialCharTok{\textasciitilde{}} \FunctionTok{dbeta}\NormalTok{(}\DecValTok{7}\NormalTok{, }\DecValTok{3}\NormalTok{)}
\NormalTok{    psi }\OtherTok{\textless{}{-}}\NormalTok{ theta }\SpecialCharTok{*}\NormalTok{ lambda }\SpecialCharTok{+}\NormalTok{ (}\DecValTok{1} \SpecialCharTok{{-}}\NormalTok{ theta) }\SpecialCharTok{*}\NormalTok{ (}\DecValTok{1} \SpecialCharTok{{-}}\NormalTok{ tau)}
\NormalTok{  \}}
\end{Highlighting}
\end{Shaded}

The model we are constructing is intended to predict which candidate is
likely to win the 2024 US presidential election (Donald Trump or
Elizabeth Warren), which is based on binomial likelihood and beta prior.
\(\theta\) is the probability that the voter claims to vote for trump
(might lying), thus, \(1 - \theta\) is the probability that the person
claims to vote for Warren. \(\lambda\) is the percentage of the
population that is actually willing to vote for trump while \(\tau\) is
the percentage of population that is actually willing to vote for
Warren. We can detect the `Quasi Liberal' and `Quasi Republican' as
\(1-\lambda\) and \(1-\tau\). To be more specific, percentage of
population that claimed they would vote for Trump/Warren but ended
voting the opposite.

Therefore, we can derive that the actual probability of a person voting
for Trump as \(\psi:\lambda\theta + (1-\tau)(1-\theta)\) and the actual
probability of a person voting for Warren as
\(\tau(1-\theta)+\theta(1-\lambda)\)

To define \(\tau\) and \(\lambda\), we would assume more Republican
people are less willing to claim their feelings in Liberal states, so
\(\tau\) \textasciitilde{} Beta(6, 4) and \(\lambda\) \textasciitilde{}
Beta(7,3)

\begin{Shaded}
\begin{Highlighting}[]
\CommentTok{\# Simulating data}
\NormalTok{n }\OtherTok{=} \DecValTok{100000}
\NormalTok{y }\OtherTok{=} \DecValTok{30000}
\NormalTok{data.JAGS }\OtherTok{=} \FunctionTok{list}\NormalTok{(}\AttributeTok{y =}\NormalTok{ y, }\AttributeTok{n =}\NormalTok{ n)}
\end{Highlighting}
\end{Shaded}

\begin{Shaded}
\begin{Highlighting}[]
\CommentTok{\# Randomly select the initial values}
\NormalTok{inits.JAGS }\OtherTok{=} \ControlFlowTok{function}\NormalTok{()\{}
  \FunctionTok{return}\NormalTok{(}\FunctionTok{list}\NormalTok{(}\AttributeTok{theta=}\FunctionTok{rbeta}\NormalTok{(}\DecValTok{1}\NormalTok{, }\DecValTok{3}\NormalTok{, }\DecValTok{7}\NormalTok{),}\AttributeTok{tau=}\FunctionTok{rbeta}\NormalTok{(}\DecValTok{1}\NormalTok{, }\DecValTok{6}\NormalTok{, }\DecValTok{4}\NormalTok{),}\AttributeTok{lambda=}\FunctionTok{rbeta}\NormalTok{(}\DecValTok{1}\NormalTok{, }\DecValTok{7}\NormalTok{, }\DecValTok{3}\NormalTok{)))}
\NormalTok{\}}
\end{Highlighting}
\end{Shaded}

\begin{Shaded}
\begin{Highlighting}[]
\CommentTok{\# Select parameters that will be simulated with MCMC model}
\NormalTok{para.JAGS }\OtherTok{=} \FunctionTok{c}\NormalTok{(}\StringTok{"theta"}\NormalTok{, }\StringTok{"tau"}\NormalTok{, }\StringTok{"lambda"}\NormalTok{, }\StringTok{"psi"}\NormalTok{)}
\end{Highlighting}
\end{Shaded}

\begin{Shaded}
\begin{Highlighting}[]
\CommentTok{\# Fit the MCMC model, number of iterations = 90000, burn in = 10000}
\NormalTok{fit.JAGS }\OtherTok{=} \FunctionTok{jags}\NormalTok{(}\AttributeTok{data=}\NormalTok{data.JAGS,}\AttributeTok{inits=}\NormalTok{inits.JAGS,}
                \AttributeTok{parameters.to.save =}\NormalTok{ para.JAGS,}
                \AttributeTok{n.chains=}\DecValTok{1}\NormalTok{,}
                \AttributeTok{n.iter=}\DecValTok{90000}\NormalTok{,}
                \AttributeTok{n.burnin=}\DecValTok{10000}\NormalTok{,}
                \AttributeTok{model.file=}\NormalTok{model.JAGS)}
\end{Highlighting}
\end{Shaded}

\begin{verbatim}
## module glm loaded
\end{verbatim}

\begin{verbatim}
## Compiling model graph
##    Resolving undeclared variables
##    Allocating nodes
## Graph information:
##    Observed stochastic nodes: 1
##    Unobserved stochastic nodes: 3
##    Total graph size: 15
## 
## Initializing model
\end{verbatim}

\begin{Shaded}
\begin{Highlighting}[]
\CommentTok{\# Print model fit summary}
\FunctionTok{print}\NormalTok{(fit.JAGS)}
\end{Highlighting}
\end{Shaded}

\begin{verbatim}
## Inference for Bugs model at "C:/Users/Ruike Xu/AppData/Local/Temp/RtmpGGHdzu/model669c5d5653cb.txt", fit using jags,
##  1 chains, each with 90000 iterations (first 10000 discarded), n.thin = 80
##  n.sims = 1000 iterations saved
##          mu.vect sd.vect   2.5%    25%    50%    75%  97.5%
## lambda     0.632   0.156  0.333  0.518  0.646  0.752  0.902
## psi        0.300   0.001  0.297  0.299  0.300  0.301  0.303
## tau        0.794   0.057  0.711  0.749  0.788  0.834  0.912
## theta      0.232   0.112  0.055  0.151  0.218  0.296  0.507
## deviance  12.718   1.370 11.791 11.878 12.163 12.926 16.780
## 
## DIC info (using the rule, pD = var(deviance)/2)
## pD = 0.9 and DIC = 13.7
## DIC is an estimate of expected predictive error (lower deviance is better).
\end{verbatim}

\begin{Shaded}
\begin{Highlighting}[]
\CommentTok{\# Generate traceplots}
\FunctionTok{traceplot}\NormalTok{(fit.JAGS,}\AttributeTok{mfrow=}\FunctionTok{c}\NormalTok{(}\DecValTok{2}\NormalTok{,}\DecValTok{3}\NormalTok{),}\AttributeTok{ask=}\ConstantTok{FALSE}\NormalTok{)}
\end{Highlighting}
\end{Shaded}

\includegraphics{STA365_homework2_code_files/figure-latex/unnamed-chunk-8-1.pdf}

\begin{Shaded}
\begin{Highlighting}[]
\CommentTok{\# Plot MCMC object to show all parameter densities along with traceplots}
\NormalTok{fit.JAGS.mcmc }\OtherTok{=} \FunctionTok{as.mcmc}\NormalTok{(fit.JAGS)}
\FunctionTok{plot}\NormalTok{(fit.JAGS.mcmc,}\AttributeTok{ask=}\ConstantTok{FALSE}\NormalTok{)}
\end{Highlighting}
\end{Shaded}

\includegraphics{STA365_homework2_code_files/figure-latex/unnamed-chunk-9-1.pdf}
\includegraphics{STA365_homework2_code_files/figure-latex/unnamed-chunk-9-2.pdf}

\begin{Shaded}
\begin{Highlighting}[]
\CommentTok{\# Summary as a MCMC object }
\FunctionTok{summary}\NormalTok{(fit.JAGS.mcmc)}
\end{Highlighting}
\end{Shaded}

\begin{verbatim}
## 
## Iterations = 10001:89921
## Thinning interval = 80 
## Number of chains = 1 
## Sample size per chain = 1000 
## 
## 1. Empirical mean and standard deviation for each variable,
##    plus standard error of the mean:
## 
##             Mean       SD  Naive SE Time-series SE
## deviance 12.7179 1.369895 4.332e-02      4.332e-02
## lambda    0.6322 0.156189 4.939e-03      2.044e-02
## psi       0.3000 0.001396 4.415e-05      4.415e-05
## tau       0.7943 0.056560 1.789e-03      9.350e-03
## theta     0.2316 0.112185 3.548e-03      1.370e-02
## 
## 2. Quantiles for each variable:
## 
##              2.5%     25%     50%     75%   97.5%
## deviance 11.79112 11.8779 12.1634 12.9260 16.7797
## lambda    0.33281  0.5179  0.6456  0.7519  0.9017
## psi       0.29718  0.2991  0.3000  0.3009  0.3028
## tau       0.71052  0.7494  0.7876  0.8338  0.9122
## theta     0.05455  0.1512  0.2180  0.2958  0.5067
\end{verbatim}

\begin{Shaded}
\begin{Highlighting}[]
\CommentTok{\# Traceplots as a MCMC object}
\FunctionTok{xyplot}\NormalTok{(fit.JAGS.mcmc,}\AttributeTok{layout=}\FunctionTok{c}\NormalTok{(}\DecValTok{2}\NormalTok{,}\DecValTok{3}\NormalTok{))}
\end{Highlighting}
\end{Shaded}

\includegraphics{STA365_homework2_code_files/figure-latex/unnamed-chunk-11-1.pdf}

\begin{Shaded}
\begin{Highlighting}[]
\CommentTok{\# Density plot as a MCMC object}
\FunctionTok{densityplot}\NormalTok{(fit.JAGS.mcmc)}
\end{Highlighting}
\end{Shaded}

\includegraphics{STA365_homework2_code_files/figure-latex/unnamed-chunk-12-1.pdf}

\hypertarget{question-3}{%
\subsection{Question 3}\label{question-3}}

\hypertarget{parta}{%
\subsubsection{Part(a)}\label{parta}}

\[\begin{aligned}
       \theta_1 = 1 + 0.065X_1 
          = 1 + 0.065 * 95 = 7.175 \\
       H_1 \sim Poisson(7.175) \\
\end{aligned}\]

\begin{Shaded}
\begin{Highlighting}[]
\CommentTok{\# Probability of Poisson distribution for at least 2 infections}
\NormalTok{P\_atleast2 }\OtherTok{\textless{}{-}} \DecValTok{1} \SpecialCharTok{{-}} \FunctionTok{ppois}\NormalTok{(}\DecValTok{1}\NormalTok{, }\FloatTok{7.175}\NormalTok{)}
\NormalTok{P\_atleast2}
\end{Highlighting}
\end{Shaded}

\begin{verbatim}
## [1] 0.9937422
\end{verbatim}

Therefore, the probability that Indonesia observes at least 2 infections
in the given time period is 0.994

\hypertarget{partb}{%
\subsubsection{Part(b)}\label{partb}}

\[\begin{aligned}
       \theta_2 = 1 + 0.065X_2 
          = 1 + 0.065 * 150 = 10.75 \\
       H_2 \sim Poisson(10.75) \\
\end{aligned}\]

\begin{Shaded}
\begin{Highlighting}[]
\CommentTok{\# Probability of Poisson distribution for at least 28 infections}
\NormalTok{P\_atleast28 }\OtherTok{\textless{}{-}} \DecValTok{1} \SpecialCharTok{{-}} \FunctionTok{ppois}\NormalTok{(}\DecValTok{27}\NormalTok{, }\FloatTok{10.75}\NormalTok{)}
\NormalTok{P\_atleast28}
\end{Highlighting}
\end{Shaded}

\begin{verbatim}
## [1] 8.378949e-06
\end{verbatim}

Therefore, the probability that Singapore observes 28 or more infections
in the given time period is approximately 0.

\end{document}
